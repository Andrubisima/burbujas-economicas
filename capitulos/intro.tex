\section{Tema a tratar, justificación y objetivos}
El tema elegido para el desarrollo del trabajo fin de grado de
administración y dirección de empresas ha sido \emph{Burbujas económicas},
justificado por mi interés en la rama de historia económica.

En el año 2.008, momento en el que me encontraba cursando la diplomatura
de ciencias empresariales en la Universidad de La Rioja, se comenzó a
escuchar el término \emph{burbuja económica}, más concretamente la burbuja
inmobiliaria que estaban sufriendo multitud de países, entre ellos
España. Muchos fueron los artículos leídos por aquel entonces acerca de
hipotecas subprime o tipos de interés entre otros muchos conceptos que
conciernen a tan complejo término. 

Por lo tanto, el presente trabajo tiene como principal objetivo dar luz
a uno de los ámbitos más ambiguos y controvertidos de la Ciencia
Económica: \emph{las burbujas económicas}. Para ello, se explicará qué son, se analizará la existencia de éstas y se ofrecerá una visión histórica a través de diferentes ejemplos surgidos a lo largo de los tiempos. Se investigarán diversas teorías sobre las mismas y se plantearán sus efectos en el panorama económico de la época. 

\section{Estructura y metodología}
La estructura del proyecto consta de cuatro capítulos con los siguientes
contenidos.

En el capítulo uno, se explicará la situación en la que el precio de un
bien o un activo, dentro de un mercado específico, se encuentra por
encima de su precio \emph{normal}. Se dará una visión histórica
recorriendo las diferentes burbujas relevantes surgidas a lo largo del
tiempo. El concepto de burbuja económica quedará ilustrado por las
primeras teorías económicas psicológicas sobre las mismas, que se basan
en factores humanos, sociales, de comportamiento y expectativas. Por
otro lado, se ofrecerán los argumentos de los autores que consideran la
inexistencia de las burbujas económicas. También se razonará si éstas
pueden generar redistribución de la riqueza o por el contrario
perjudica gravemente la economía. 

En el segundo capítulo, se describe la que posiblemente sea la primera
burbuja económica de la historia, la tulipomanía, surgida en los Países
Bajos en 1.637. A lo largo del capítulo, se da una amplia visión del
contexto histórico holandés en el que se desarrolló dicha burbuja, se
analizan las fases de ésta, se nombran diferentes fuentes de las que se
pueden obtener datos sobre los precios de los tulipanes en la época y
por último se estudia la influencia de las redes sociales del momento y
cómo afectó al fenómeno especulativo. 

En el Capítulo 3, se expone la burbuja de la Compañía del Mississippi
desarrollada en Francia a lo largo del siglo XVIII. Se explica la
implantación del revolucionario sistema económico financiero
desarrollado por el escocés John Law. Se analizará como afectó este
método a la deuda del país, como devastó el sistema financiero francés
desarrollando una burbuja económica y se dan diferentes conclusiones
de autores. 

En el Capítulo 4, se estudia la burbuja económica de la Compañía de los
Mares del Sur dada en Inglaterra en el siglo XVIII. Se da una visión
del marco histórico inglés y su paralelismo con la burbuja de la
Compañía del Mississippi, de la que le separan escasos años de
diferencia. 

Por último, se han extraído una serie de conclusiones alcanzadas por los
autores mencionados a lo largo del proyecto y las cuales se han ido
encuadrando dentro del tema a tratar. Además, a lo largo del trabajo se
han ido incluyendo conclusiones que, de forma parcial, se presentan en
algunos epígrafes.

La metodología llevada a cabo para conceptualizar el tema a desarrollar
a lo largo del proyecto ha sido la recopilación de artículos de
investigación, lectura y comprensión de textos para poder ofrecer el mayor rigor posible. 
