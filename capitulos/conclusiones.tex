Tras haber profundizado en el concepto de “burbuja económica”, a continuación se presentan las conclusiones que se han extraído a lo largo de este proyecto. 

En vista de la variedad de definiciones propuestas por los autores estudiados, queda patente la complejidad y dificultad que existe a la hora establecer una definición única. Finalmente, se ha concluido que la definición más adecuada para describir qué es una burbuja económica es la de DeMarzo: “Se define una situación de burbuja en el momento en el que existe un precio de mercado de un activo que es superior a su valor fundamental, siendo el valor fundamental el valor presente de los pagos futuros”.

Gracias al análisis que se ha realizado a lo largo del trabajo, se ha podido obtener una idea general de los orígenes, formación, efectos e impactos que una burbuja económica puede tener en la sociedad. Todo ello se ha podido observar teórica y gráficamente a través de las principales burbujas económicas acontecidas a lo largo de la historia. Otro aspecto importante a tener en cuenta y que se ha puesto de manifiesto es establecer una serie de variables que influyen en el desarrollo de las burbujas, teniendo en cuenta la naturaleza de éstas y su posterior evolución.

Se ha observado también que una burbuja podría provocar un efecto general positivo o por el contrario un efecto negativo. No obstante, en cualquier caso, las burbujas económicas generan una redistribución de la riqueza. Esta distribución no sólo se concibe entre los agentes participantes si no que afecta al resto de agentes que en principio no estaban involucrados, al provocar importantes daños colaterales.

Algunos de los factores que propician las burbujas son: el aumento de los precios, la masiva inversión y especulación, la existencia de una empresa importante envuelta de incertidumbre, el apalancamiento y las acciones que los gobiernos llevan a cabo. También se ha mostrado gráfica y teóricamente que las burbujas siguen un patrón de precios determinado por cuatro fases.

Como se ha comprobado, las burbujas económicas hacen su aparición en diferentes tipos de activos, pero cabe señalar que en el contexto moderno actual es probable que surjan principalmente en el mercado de valores y en el mercado inmobiliario.

En el caso de la tulipomanía, cabe destacar que la confianza en el mercado holandés se sustentaba en dos variables particularmente importantes: las interacciones comerciales con un alto grado de incertidumbre y la especulación\footnote{Entonces la especulación no estaba considerada como tal ya que se mantenía la esperanza de que esta situación de mercado se mantuviese de por vida.}. Los autores contemporáneos atribuyen a los especuladores de la tulipomanía la mayor parte de culpa sobre el desastre ocurrido, pero lo que realmente los hizo influyentes en el mercado fue que socialmente no se distinguían de los comerciantes entendidos de tulipanes. Por este motivo, el impacto de la burbuja holandesa en 1.637, fue una amenaza en los pilares de los lazos sociales establecidos en el país. Gran parte del éxito comercial se basó en estas relaciones sociales, variable muy importante a la hora de estudiar la tulipomanía en todo su conjunto.

De la burbuja económica de la Compañía del Mississippi, cabe destacar que la ambición de John Law se correspondía con la íntegra transformación de las finanzas públicas francesas a través de dos innovaciones radicales: la sustitución de la moneda de metal con el dinero fiduciario y la sustitución de la deuda pública por acciones. La Compañía quebró no por pertenecer a un mercado frenético e irracional, sino por la influencia ejercida por John Law dentro del mercado.

Se ha de incidir en que la Compañía de los Mares del Sur se desarrolló con gran éxito porque la población tenía especial interés por los juegos de azar. Se alcanzó un momento en el que era muy semejante el funcionamiento de estos juegos de azar con la forma en que se realizaban las suscripciones de la Compañía, es decir, a ciegas. Cada uno de los sucesivos niveles de emisión de acciones alteraba a la ciudadanía y contagiaba la codicia entre los inversores y la población. Ciertamente, el comportamiento de los inversores en ese espacio temporal parece reflejar lo observado en los estudios recientes de los mercados de juegos de azar. Además se tiene que tener en cuenta que el comportamiento del inversor puede llegar a ser maníaco e irracional. Es destacable de este último episodio, el relativamente escaso tiempo de duración con respecto a las otras dos burbujas estudiadas.

Se ha de pensar que no es posible la comparación entre los métodos de investigación de mercado utilizados por los analistas de la época en que surgieron las tres burbujas analizadas, con los que se utilizan hoy en día ya que en la actualidad éstos se dedican en su mayoría a convencer a los inversores de que sólo hay una dirección en la que colocar sus acciones. En cambio, en los siglos XVII y XVIII, los analistas estaban camuflados entre la multitud de compradores y vendedores.

A la vista de todo lo expuesto en este trabajo, a la hora de tomar decisiones comerciales, sea el agente que fuere, se ha de plantear una cuestión: “¿A dónde se quiere llegar con este movimiento económico y por qué se realiza?” Hoy en día, el ansia de poder hace que se pierdan los valores y las prioridades que como persona y agente se han de poseer. La pérdida de valores se encuentra, en numerosas ocasiones, cegada por la avaricia. Lo que conlleva, en muchas ocasiones, a la toma de decisiones irracionales debido a la premura con que se producen. Por lo tanto, si no tenemos clara la respuesta a la cuestión anteriormente planteada, puede ocurrir lo contrario de lo que se esperaba. 
